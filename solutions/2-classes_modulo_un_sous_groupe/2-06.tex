\begin{enumerate}
  \item % a)
    Soit $(a,b)\in\R^2$.
    Pour tout $x\in\R$, on a
    \begin{gather*}
      \sigma_a^2(x) = \sigma(a - x) = a - (a - x) = x, \\
      (\sigma_b\circ \tau_a)(x) = \sigma_b(x + a) = b - x - a = \tau_{-a}(\sigma_b(x)),
    \end{gather*}
    donc $\sigma_a^2 = \id_\R$ et $\sigma_b\circ \tau_a = \tau_{-a}\circ \sigma_b$.

  \item % b)
    Soit $a\in\R$.
    Pour tout $x\in\R$, on a
    \[
      \frac{x + \sigma_a(x)}{2} = \frac{a}{2},
    \]
    donc $\sigma_a$ est la symétrie par rapport au point $a/2$.

  \item % c)
    \emph{Isométries de $\R$.}
    Soit $f$ une isométrie de $\R$.
    Pour tout réel $x$, on a $\abs{f(x) - f(0)} = \abs{x}$, d'où
    \[
      f(x) = f(0) + x \quad\text{ou}\quad f(x) = f(0) - x.
    \]
    Supposons qu'il existe $(x,y)\in\R^2$ tel que $x\neq y$,
    \[
      f(x) = f(0) + x \quad\text{et}\quad f(y) = f(0) - y.
    \]
    Par soustraction, nous obtenons
    \[
      f(x) - f(y) = x + y.
    \]
    Comme $f$ est une isométrie, nous avons aussi $\abs{f(x) - f(y)} = \abs{x - y}$, d'où
    \[
      x + y = x - y \quad\text{ou}\quad x + y = -x + y,
    \]
    soit
    \[
      y = 0 \quad\text{ou}\quad x = 0.
    \]
    Nous avons montré que
    \begin{gather*}
      \forall x\in\R, f(x) = f(0) + x \\
      \shortintertext{ou}
      \forall x\in\R, f(x) = f(0) - x.
    \end{gather*}
    En d'autres termes, $f = \tau_{f(0)}$ ou $f = \sigma_{f(0)}$, ou encore, une isométrie de $\R$ est soit une translation, soit une symétrie par rapport à un point.

    \emph{$\mathcal{I}(1)$ est un sous-groupe non abélien de $\Sym{\R}$.}
    On vérifie sans peine que les symétries et les translations sont des bijections, donc $\mathcal{I}(1)\subseteq\Sym{\R}$.
    De plus, $\mathcal{I}(1)$ est non vide et pour tout $(f,g)\in\mathcal{I}(1)$, on a $f\circ g^{-1}\in\mathcal{I}(1)$.
    Par conséquent, $\mathcal{I}(1)$ est un sous-groupe de $\Sym{\R}$.
    Enfin, si $(a,b)\in\R^2$ alors $\sigma_b\circ\tau_a = \tau_{-a}\circ\sigma_b$.
    Or $\tau_{-a}\neq \tau_a$ si $a\neq 0$.
    Il s'ensuit que $\sigma_b\circ\tau_a \neq \tau_a\circ\sigma_b$ si $a\neq 0$, ce qui prouve que le sous-groupe $\mathcal{I}(1)$ est non abélien.

  \item % d)
    Soient $a$ et $b$ deux réels.
    Les translations appartiennent à la même classe à droite modulo $T$; en effet,
    \[
      \tau_a \circ \tau_b^{-1} = \tau_a \circ \tau_{-b} = \tau_{a - b} \in T.
    \]
    De même, les symétries appartiennent à la même classe à droite modulo $T$, car
    \[
      \sigma_a \circ \sigma_b^{-1} = \sigma_a \circ \sigma_b = \tau_{a - b} \in T.
    \]
    En revanche,
    \[
      \tau_a \circ \sigma_b^{-1} = \tau_a \circ \sigma_b = \sigma_{a + b} \notin T
    \]
    montre que les translations et les symétries appartiennent à des classes distinctes.
    Puisqu'une isométrie de $\R$ est soit une translation, soit une symétrie, nous en déduisons qu'il y a exactement deux classes à droites modulo $T$ dans $\mathcal{I}(1)$, d'où $[\mathcal{I}(1):T] = 2$.
\end{enumerate}
