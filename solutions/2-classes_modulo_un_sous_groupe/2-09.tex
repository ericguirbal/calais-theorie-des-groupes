Suivons l'indication de l'énoncé et considérons la correspondance
%
\begin{align*}
  \varphi\colon\clsd*{G}{H} &\longrightarrow\clsd*{G}{H} \\
                        Kx &\longmapsto K'gx.
\end{align*}
%
\begin{itemize}
  \item 
    Montrons que $\varphi$ est une application. Supposons $Kx = Kx'$,
    c'est-à-dire $xx'^{-1}\in K$. Alors 
    $gx(gx')^{-1} = gxx'^{-1}g^{-1} \in gKg^{-1} = K'$ donc $K'gx=K'gx'$,
    c'est-à-dire $\varphi(Kx)=\varphi(Kx')$. 

  \item
    $\varphi$ est injective. En effet,
    %
    \begin{align*} 
      \varphi(Kx) = \varphi(Kx')
        &\iff K'gx = K'gx'\\ 
        &\iff (gx)(gx')^{-1} \in K'\\ 
        &\iff gxx'^{-1}g^{-1} \in K'\\ 
        &\iff xx'^{-1} \in g^{-1}K'g = K\\ 
        &\iff Kx = Kx' 
    \end{align*} 
    %
  \item 
    $\varphi$ est surjective. En effet, quel que soit
    $K'x' \in \clsd*{G}{K'}$, on peut écrire
    $\varphi(Kg^{-1}x') = K'g(g^{-1}x') = K'x'$.
\end{itemize}

Nous avons démontré que $\clsd*{G}{K}$ et $\clsd*{G}{K'}$ sont équipotents, d'où
l'égalité $[G:K]=[G:K']$.
