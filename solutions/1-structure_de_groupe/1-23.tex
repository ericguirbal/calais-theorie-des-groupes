\begin{enumerate}
  \item
    Les ensembles $\Gamma_1$, $\Gamma_2$ et $\Gamma_3$ sont non vides et $\Gamma_1\subseteq\GL(2,\R)$, $\Gamma_2\subset \C^*$ et $\Gamma_3\subseteq\Zn{5}$.
    Pour vérifier la stabilité par rapport à la multiplication et à l'inverse, construisons leurs tables de Cayley.

    Posons $I = \matrice{1 & 0 \\ 0 & 1}$, $A = \matrice{0 & 1 \\ -1 & 0}$, $B = \matrice{-1 & 0 \\ 0 & -1}$ et $C = \matrice{0 & -1 \\ 1 & 0}$ de sorte que $\Gamma_1 = \set{I, A, B, C}$.
    Sa table de Cayley est :
    \begin{center}
      \begin{tabular}{r|rrrr}
        $\times$ & $I$ & $A$ & $B$ & $C$ \\
        \midrule
             $I$ & $I$ & $A$ & $B$ & $C$ \\
             $A$ & $A$ & $B$ & $C$ & $I$ \\
             $B$ & $B$ & $C$ & $I$ & $A$ \\
             $C$ & $C$ & $I$ & $A$ & $B$ \\
      \end{tabular}
    \end{center}

  Les tables de Cayley de $\Gamma_2$ et $\Gamma_3$ sont les suivantes :

  \begin{minipage}{0.49\textwidth}
    \centering
    \begin{tabular}{r|rrrr}
      $\times$ & $1$   & $\I$  & $-1$  & $-\I$ \\
      \midrule
        $1$    & $1$   & $\I$  & $-1$  & $-\I$ \\
       $\I$    & $\I$  & $-1$  & $-\I$ & $1$   \\
       $-1$    & $-1$  & $-\I$ & $1$   & $\I$  \\
      $-\I$    & $-\I$ & $1$   & $\I$  & $-1$  \\
    \end{tabular}
  \end{minipage}
  %
  \begin{minipage}{0.49\textwidth}
    \centering
    \begin{tabular}{r|rrrr}
      $\times$ & $\cl{1}$ & $\cl{2}$ & $\cl{3}$ & $\cl{4}$ \\
      \midrule
      $\cl{1}$ & $\cl{1}$ & $\cl{2}$ & $\cl{3}$ & $\cl{4}$ \\
      $\cl{2}$ & $\cl{2}$ & $\cl{4}$ & $\cl{1}$ & $\cl{3}$ \\
      $\cl{3}$ & $\cl{3}$ & $\cl{1}$ & $\cl{4}$ & $\cl{2}$ \\
      $\cl{4}$ & $\cl{4}$ & $\cl{3}$ & $\cl{2}$ & $\cl{1}$ \\
    \end{tabular}
  \end{minipage}

  Nous en déduisons que $\Gamma_1$ est un sous-groupe de $\GL(2,\R)$, $\Gamma_2$ un sous-groupe de $(\C^*,\times)$ et $\Gamma_3$ un sous-groupe de $\Zn{5}$.

\item
  En identifiant chaque élément du groupe $\Gamma_1$ à son image dans le groupe $\Gamma_2$ par la bijection $f\from\Gamma_1\to\Gamma_2$ définie par $f(I) = 1$, $f(A) = \I$, $f(B) = -1$ et $f(C) = -\I$, nous constatons que la table de Cayley du groupe $\Gamma_1$ est la même que celle du groupe $\Gamma_2$.
  Nous en déduisons que les groupes $\Gamma_1$ et $\Gamma_2$ sont isomorphes.

  De même, on montre que les groupes $\Gamma_1$ et $\Gamma_3$ sont isomorphes à l'aide de la bijection $g\from\Gamma_1\to\Gamma_3$ définie par $g(I) = \cl{1}$, $g(A) = \cl(2)$, $g(B) = \cl{4}$ et $g(C) = \cl{3}$.

  De ce qui précède, nous déduisons que $\Gamma_1$ est isomorphe à $\Gamma_3$ (remarque~1.71).

  Remarquons que $\Gamma_1 = \Gr{A}$, $\Gamma_2 = \Gr{\I}$ et $\Gamma_3 = \Gr{\cl{2}}$, donc les groupes $\Gamma_1$, $\Gamma_2$ et $\Gamma_3$ sont cycliques.
\end{enumerate}
