\begin{enumerate}
  \item
    \emph{La correspondance $\mu$ est une application.} 

    Soient $(x,x',y,y')\in\Z^4$. 
    L'identité
    \[
      x'y' - xy = x'(y' - y) + y(x' - x)
    \]
    montre que si $n$ divise $x' - x$ et $y' - y$, alors $n$ divise $x'y' - xy$. 
    Autrement dit, si $\cl{x} = \cl{x'}$ et $\cl{y} = \cl{y'}$, alors $\cl{xy} = \cl{x'y'}$.

    \emph{$\Zn{n}$ est un anneau unitaire commutatif.}

    La multiplication dans $\Zn{n}$ est associative: quels que soient $\cl{x}$, $\cl{y}$, $\cl{z}$ dans $\Zn{n}$, on a
    \[
      (\,\cl{x}\;\cl{y}\,)\,\cl{z} 
        = \cl{xy}\;\cl{z}
        = \cl{(xy)z}
        = \cl{x(yz)}
        = \cl{x}\;\cl{yz}
        = \cl{x}\,(\,\cl{y}\;\cl{z}\,).
    \]

    La multiplication est commutative: quels que soient $\cl{x}$, $\cl{y}$ dans $\Zn{n}$, on a
    \[
      \cl{x}\;\cl{y} = \cl{xy} = \cl{yx} = \cl{y}\;\cl{x}.
    \]

    La multiplication est distributive par rapport à l'addition: quels que soient $\cl{x}$, $\cl{y}$, $\cl{z}$ dans $\Zn{n}$, on a
    \[
      \cl{x}\,(\,\cl{y} + \cl{z}\,)
        = \cl{x}\;\cl{y + z}
        = \cl{x(y + z)}
        = \cl{xy + xz}
        = \cl{xy} + \cl{xz}
        = \cl{x}\;\cl{y} + \cl{x}\;\cl{y}.
    \]

    La multiplication admet un élément neutre: quel que soit $\cl{x}$ dans $\Zn{n}$, on a
    \[
      \cl{x}\;\cl{1} = \cl{x\times 1} = \cl{x}.
    \]

  \item
    L'ensemble $G_p$ est fini et nous avons déjà démontré que la multiplication est associative.
    Soient $\cl{a}$, $\cl{x}$ et $\cl{y}$ trois éléments de $G_p$ tels que $\cl{a}\;\cl{x} = \cl{a}\;\cl{y}$. 
    Alors $\cl{ax} = \cl{ay}$, donc $p$ divise $a(x-y)$.
    Or $p$ ne divise pas $a$ puisque $\cl{a}\neq\cl{0}$, donc $p$ divise $x - y$ ce qui signifie que $\cl{x} = \cl{y}$.
    Nous en déduisons que la multiplication est simplifiable à gauche. 
    Elle est commutative, donc elle est également simplifiable à droite.
    D'après le résultat de l'exercice~4, l'ensemble $G_p$ muni de la multiplication définie dans $\Zn{p}$ est un groupe.
    Par conséquent tout élément non nul de $\Zn{p}$ est inversible, d'où nous concluons que $\Zn{p}$ est un corps.

  \item
    Supposons $n$ non premier. 
    Il existe deux entiers $k > 1$ et $l > 1$ tels que $n = kl$. 
    Alors $\cl{k}\;\cl{l} = \cl{n} = \cl{0}$  avec $\cl{k}\neq \cl{0}$ et $\cl{l}\neq \cl{0}$, donc $\cl{k}$ n'est pas inversible : $\Zn{n}$ n'est pas un corps.
\end{enumerate}
