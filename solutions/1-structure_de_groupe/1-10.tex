\begin{enumerate}[a)]
  \item $\Q_p$ est un sous-groupe de $(\Q,+)$, en effet pour tout
    $(a,n)\in\Z\times\N$, on a
    \[
      \frac{a}{p^n}-\frac{b}{p^m} = \frac{ap^m-bp^n}{p^{n+m}}
    \]
    avec $ap^m-bp^n\in\Z$ et $n+m\in\N$ donc
    $\dfrac{a}{p^n}-\dfrac{b}{p^m}\in\Q_p$.

    Ensuite
    \[
      \Q_p = \bigcup_{n\in\N} \left\{\frac{a}{p^n}; a\in\Z\right\} =
        \bigcup_{n\in\N} \,\langle\frac{1}{p^n}\rangle
    \]

  \item $\varphi$ est clairement injective, et de l'égalité
    $\frac{a}{p^n} = p\frac{a}{p^{n+1}}$ on déduit qu'elle est également
    surjective et donc que $\varphi$ est une permutation de $\Q_p$.

    Pour tout $(x,y)\in\Q_p^2$, on a
    $\varphi(x+y) = p(x+y) = px+py = \varphi(x)+\varphi(y)$. Or un homomorphisme
    bijectif est un isomorphisme (Proposition 1.66) donc $\varphi$ est un
    automorphisme du groupe $(\Q_p,+)$.
\end{enumerate}
