Soit $G_1$ et $G_2$ deux groupes isomorphes et $f\colon G_1\to G_2$ un
isomorphisme.
%
\begin{enumerate}[a)]
\item On définit un morphisme de groupes $f_\sharp\colon\Aut(G_1)\to\Aut(G_2)$
en posant $f_\sharp(\psi) = f\circ\psi\circ f^{-1}$. En effet pour tout
$\psi_1,\psi_2\in\Aut(G_1)$ on a:
%
\begin{align*}
f_\sharp(\psi_1\circ\psi_2) &= f\circ(\psi_1\circ\psi_2)\circ f^{-1} \\
  &= (f\circ\psi_1\circ f^{-1})\circ(f\circ\psi_2\circ f^{-1}) \\
  &= f_\sharp(\psi_1)\circ f_\sharp(\psi_2)
\end{align*}

De plus si $\psi\in\Aut(G_1)$,
%
\begin{align*}
f_\sharp(\psi) = \id_{G_2}
  &\iff f\circ\psi\circ f^{-1} = \id_{G_2} \\
  &\iff \psi = f^{-1}\circ\id_{G_2}\circ f \\
  &\iff \psi = \id_{G_2}
\end{align*}
%
et si $\psi\in\Aut(G_2)$, on a $f^{-1}\circ\psi\circ f\in\Aut(G_1)$ et
\[
  f_\sharp(f^{-1}\circ\psi\circ f) = f\circ(f^{-1}\circ\psi\circ f)\circ
  f^{-1} = \id_{G_2}\circ\psi\circ\id_{G_2} = \psi
\]
Donc $f_\sharp$ est un isomorphisme de groupes et $\Aut(G_1)\iso\Aut(G_2)$.

\item Si $\sigma_g\in\Int(G_1)$, alors pour tout $x\in G_2$,
%
\begin{align*}
[f_\sharp(\sigma_g)](x)& = (f\circ\sigma_g\circ f^{-1})(x) \\
  & = f(gf^{-1}(x)g^{-1}) \\
  & = f(g)xf(g)^{-1} \\
  & = \sigma_{f(g)}(x)
\end{align*}
%
c'est-à-dire $f_\sharp(\sigma_g) = \sigma_{f(g)}\in\Aut(G_2)$ ou encore
$f_\sharp(\Int(G_1))\subset\Int(G_2)$. Ainsi on définit un homomorphisme de
groupes $f_*\colon\Int(G_1)\to\Int(G_2)$ en posant
$f_* = f_\sharp\vert^{\Int(G_2)}_{\Int(G_1)}$. Il est injectif (comme
$f_\sharp$) et est aussi surjectif, en effet si $\sigma_g\in\Int(G_2)$, alors
$f_*(\sigma_{f^{-1}(g)}) = \sigma_{f(f^{-1}(g))} = \sigma_g$. Donc $f_*$ est
un isomorphisme de groupes et $\Int(G_1)\iso\Int(G_2)$.
\end{enumerate}
