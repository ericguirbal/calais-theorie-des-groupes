Quel que soit $x\in\R^*$, notons $M_x$ la matrice $\mat{x & x \\ 0 & 0}$.
La multiplication des matrices s'écrit
\[
  M_xM_y = M_{xy} = M_{yx} = M_yM_x
\]
pour tout $(x,y)\in\R^{*2}$. Il s'ensuit que la multiplication des matrices
définit une loi interne commutative sur l'ensemble $\Gamma$. Nous savons déjà
que cette loi est associative.  Elle admet pour élément neutre la matrice $M_1$,
de plus toute matrice $M_x$ est inversible d'inverse $M_{x^{-1}}$, en effet $M_x
M_{x^{-1}} = M_{x x^{-1}} = M_1$.  Nous avons montré que l'ensemble $\Gamma$ 
muni de la multiplication des matrices est un groupe abélien.

Le groupe $\Gamma$ n'est pas un sous-groupe de $\GL(2,\R)$; en effet, les
matrices $M_x$ ont un déterminant nul et donc ne sont pas inversibles dans
$\GL(2,\R)$.

Montrons que les groupes $\Gamma$ et $(R^*,\times)$ sont isomorphes.
Soit l'application $\varphi\colon\Gamma\rightarrow\R^*$ définie par
$\varphi(M_x) = x$. L'application $\varphi$ est un morphisme de groupes. En
effet, pour tout $(M_x,M_y)\in\Gamma^2$, on a
\[
  \varphi(M_x M_y) = \varphi(M_{xy}) = xy = \varphi(M_x)\varphi(M_y).
\]
Le morphisme $\varphi$ est injectif, car $\varphi(M_x) = 1$ implique $x=1$
c'est-à-dire $M_x = M_1$. Enfin, le morphisme $\varphi$ est surjectif, car pour
tout $x\in\R^*$, on a $\varphi(M_x) = x$. On conclut que $\varphi$ est un
isomorphisme de groupes, donc que les groupes $\Gamma$ et $(\R^*,\times)$ sont
isomorphes.

