Soit $G$ un groupe d'ordre $2n$. On définit sur $G$ une relation d'équivalence
$\mathcal{R}$ ainsi:
\[
  x,y\in G,\quad x \mathcal{R} y \Leftrightarrow x = y \text{ ou } x = y^{-1}.
\]
La classe d'équivalence d'un élément $x\in G$ est $\cl{x} = \{ x, x^{-1} \}$.
Soit $\{ x_i \}_{i\in I}$ une famille de réprésentants des classes distinctes
modulo $\mathcal{R}$.  On a $1 \leq |\cl{x_i}| \leq 2$. On note $k$ le nombre
de classes de cardinal 1 et $l$ le nombre de classes de cardinal 2. De
$\sum_{i\in I} |\cl{x_i}| = 2n$ on tire $k+2l = 2n$ puis $k = 2(n-l)$,
c'est-à-dire que le nombre d'éléments $x\in G$ tel que $x^2=e$ est
nécessairement pair. Comme $e$ est l'un d'eux, il existe au moins un $x\in G$
distinct de $e$ tel que $x^2 = e$.

