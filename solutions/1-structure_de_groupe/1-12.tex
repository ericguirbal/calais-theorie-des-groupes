L'énoncé comporte une petite erreur de frappe.
La définition correcte de l'ensemble $\Gamma_\infty$ est
\[
  \Gamma_\infty = \set{z\in\C\given\exists n\in\N^*, z^n = 1}.
\]
Il est clair que $\Gamma_\infty$ est une partie non vide ($1\in\Gamma_\infty$) de $\C^*$.
Soit $(z,w)\in\Gamma^2_\infty$.
Il existe $(n,m)\in\N^{*2}$ tel que $z^n = 1$ et $w^m = 1$.
On a alors $nm\neq 0$ et $(zw^{-1})^{nm} = (z^n)^m (w^m)^{-n} = 1$, donc $zw^{-1}\in\Gamma_\infty$.
Le théorème~1.24 permet de conclure que $\Gamma_\infty$ est un sous-groupe de $(\C^*,\times)$.

\begin{remarque}
  Notons que $\Gamma_\infty = \bigcup_{n \geq 1} U_n$ où $U_n$ est le sous-groupe des racines $n$-ième de l'unité (exemple~1.30).
  Or nous savons que l'union d'une famille de sous-groupes n'est, en général, pas un sous-groupe (exemple~1.26).
  La famille $\set{U_n}_{n\geq 1}$ n'étant pas totalement ordonnée pour l'inclusion (par exemple, $U_2$ et $U_3$ ne sont pas comparables), la proposition~1.27 n'est pas applicable.
  En revanche, on vérifie facilement que pour tous entiers $n\geq 1$ et $m\geq 1$, on a $U_n\subseteq U_{mn}$ et $U_m\subseteq U_{mn}$.
  On dit que l'ensemble ordonné $(\set{U_n}_{n\geq 1},\subseteq)$ est \emph{filtrant à droite}.
  Cet exemple suggère une extension de la proposition~1.27 :

  \begin{proposition}
    Soient $G$ un groupe et $\set{H_i}_{i\in I}$ une famille de sous-groupes de $G$ ordonnée par l'inclusion.
    Si pour tout $(i, j)\in I^2$, il existe $k\in I$ tel que $H_i\subseteq H_k$ et $H_j\subseteq H_k$, alors $\bigcup_{i\in I} H_i$ est un sous-groupe de $G$.
  \end{proposition}
\end{remarque}

