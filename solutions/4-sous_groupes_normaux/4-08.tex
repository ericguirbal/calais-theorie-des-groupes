Nous savons déjà que $\Alt{3}\normal\Sym{3}$ (théorème 4.7).  Soit $H\neq (e)$
un sous-groupe propre normal de $\Sym{3}$. Alors, d'après le théorème de
Lagrange, $\ordre(H)\in\set{2,3}$.  Supposons que $\ordre(H) = 2$, alors
$H = \Gr{\tau}$ où $\tau$ est une transposition.  Notons $\tau = (i,j)$, et
soit la transposition $\gamma = (i,k)$ où $k\in\N_3\setminus\set{i,j}$.  Alors
$\gamma\circ\tau\circ\gamma^{-1} = (j,k) \notin H$, donc $H\notnormal\sym{3}$.
On en déduit que $\ordre(H) = 3$.  Le sous-groupe $H$ est cyclique (car d'ordre
premier) et est donc engendré par l'un des cycles $\sigma_1 = (1,2,3)$ ou
$\sigma_2 = (1,3,2)$. On remarque que $\sigma_2 = \sigma_1^2$ et que
$\sigma_1\in\Alt{3}$ donc $H = \Gr{\sigma_1}\subseteq\Alt{3}$. Comme
$\card{\Alt{3}} = 3!/2 = 3$, il vient $H = \Alt{3}$. En résumé, $\Alt{3}$ est
l'unique sous-groupe propre normal de $\Sym{3}$ différent de $(e)$.
