Soient les décompositions canoniques des permutations $u$ et $v$ en produit de cycles disjoints, $u = (1,2,5)(4,6)$ et $v = (1,5)(2,3,4)$, et soit $\sigma\in\Sym{6}$ telle que $(\sigma(1),\sigma(2),\sigma(5)) = (2,3,4)$ et $(\sigma(4),\sigma(6)) = (1,5)$, par exemple
\[
  \sigma = \matrice{1 & 2 & 3 & 4 & 5 & 6 \\ 2 & 3 & 6 & 1 & 4 & 5}.
\]
Alors en utilisant le résultat de la question a) de l'exercice III.24, on trouve
%
\begin{align*}
    \sigma\circ u\circ\sigma^{-1} 
        &= \sigma\circ (1,2,5)\circ\sigma^{-1}\circ\sigma\circ (4,6)\circ\sigma^{-1} \\
        &= (\sigma(1),\sigma(2),\sigma(5))(\sigma(4),\sigma(6)) \\
        &= (2,3,4) (1,5) \\
        &= v,
\end{align*}
%
ainsi les permutations $u$ et $v$ sont conjuguées dans $\Sym{6}$.
