%%% Exercice 4.16

Démontrons par récurrence forte sur l'entier $n$ l'assertion suivante est vraie pour tout $n \geq 2$ :
\begin{quote}\itshape
  Soit $G$ un groupe abélien d'ordre $n$.
  Si $p$ est un diviseur premier de $n$, alors $G$ contient un élément d'ordre $p$.
\end{quote}

Soit $G$ est un groupe d'ordre $2$.
L'unique diviseur premier de $2$ est $2$, or l'unique élément $x$ distinct de l'élément neutre est d'ordre $2$, donc l'assertion est vraie pour $n = 2$.

Soit un entier $k > 1$.
Supposons que l'assertion est vraie pour tout entier $n$ tel que $2\leq n\leq k$.
Soient $G$ un groupe d'ordre $k + 1$ et $p$ un diviseur premier de $k + 1$.
Distinguons deux cas.

\textit{Cas 1 : le groupe $G$ n'a pas d'autres sous-groupes que $G$ et $(e)$.}
Dans ce cas, $G$ est cyclique d'ordre premier (proposition~3.16), donc tout élément de $G$ distinct de $e$ est d'ordre $p$ (proposition~3.18).

\textit{Cas 2 : le groupe $G$ admet un sous-groupe propre $H\neq (e)$.}
Comme $G$ est abélien, le sous-groupe $H$ est normal dans $G$ et $\grq{G}{H}$ est un groupe abélien (proposition~2.24).
De $\ordre(G) = \ordre\bigl(\grq{G}{H}\bigr)\ordre(H)$ (proposition~2.15), on déduit que $p$ divise $\ordre\bigl(\grq{G}{H}\bigr)$ ou $\ordre(H)$.
Supposons que $p$ divise $\ordre(H)$.
Étant donné que $2\leq\ordre(H)\leq n$, d'après l'hypothèse de récurrence, il existe dans $H$ un élément d'ordre $p$.
Supposons que $p$ divise $\ordre\bigl(\grq{G}{H}\bigr)$.
Comme $2\leq\ordre\bigl(\grq{G}{H}\bigr)\leq n$, d'après l'hypothèse de récurrence, le groupe $\grq{G}{H}$ contient un élément d'ordre $p$ et celui-ci possède un représentant d'ordre $p$ dans $G$ (exercice~4.15).

Dans les deux cas, nous avons montré que l'assertion est vraie pour $n = k + 1$.
Le principe de la récurrence forte nous permet alors de conclure que l'assertion est vraie pour tout entier $n\geq 2$.
