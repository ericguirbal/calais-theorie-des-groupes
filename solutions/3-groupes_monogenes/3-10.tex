%%% Exercice 3.10

\begin{enumerate}
  \item % a)
    Les groupes cycliques étant abéliens, on a $(ab)^r = a^r b^r$ pour tout $(a, b)\in C_n^2$, donc $f\in\End(C_n)$.
    D'après le corollaire~3.7, on a $\ker f = \set{a\in C_n\given\text{$\ordre(a)$ divise $r$}}$.
    Des équivalences
    \[
      x^l\in\ker f
        \iff rl\in n\Z
        \iff l\in s\Z,
    \]
    nous déduisons que $a\in\ker f$ si et seulement s'il existe un entier $k$ tel que $a = x^{ks}$ ; autrement dit, $\ker f = \set{x^{ks} \given k\in\Z}$.
    Or, quel que soit $(k, k')\in\Z^2$,
    \[
      x^{ks} = x^{k's} \iff ks\equiv k's \pmod{n} \iff k\equiv k' \pmod{r},
    \]
    si bien que $\ker f = \set{x^{ks}\given k\in\N_r}$ ; il s'ensuit que $\ordre(\ker f) = r$.
    À l'aide du 1\ier~théorème d'isomorphisme et de la proposition~2.15, on conclut que $\ordre(\im f) = \ordre(C_n)/\ordre(\ker f) = s$.

  \item % b)
    Pour tout $a\in C_n$, on a $(h\circ f)(a) = (a^r)^s = a^n = e$, donc $\im f \subseteq\ker h$.
    De la question précédente, nous déduisons que $\ordre(\ker h) = s$ et  $\ordre(\im f) = s$ ;  il s'ensuit que $\im f = \ker h$.
    Par symétrie, nous avons également $\im h = \ker f$.
\end{enumerate}
