%%% Exercice 3.30

\begin{enumerate}
  \item % a)
    Soit $u$ un élément de $U$.
    Par définition de l'ordre d'un élément, $\card{\Gr{u}} = \ordre(u) = 2$, donc d'après le théorème de Lagrange $\card{\Gr{u}}$ divise $\card{G}$ ;
    il s'ensuit que $G$ est d'ordre pair.

  \item % b)
    De $u^2 = e$, on déduit que $u = u^{-1}$.
    De même $v = v^{-1}$.

    Étant donné que $u^2 = v^2 = e$, $u^{-1} = u$, $v^{-1} = v$, les éléments de $\Gr{u, v}$ distincts de $e$ s'écrivent
    \[
      x_1x_2\dots x_n
    \]
    où $n\in\N^*$, $x_i\in\set{u, v}$ pour tout $i\in\set{1, 2, \dots, n}$ et $x_i\neq x_{i + 1}$ pour tout $i\in\set{1, \dots, n - 1}$.

    Distinguons deux cas selon la parité de $n > 0$.
    \begin{itemize}
      \item
        \emph{Cas $n$ pair.}
        Soit $k\in N^*$ tel que $n = 2k$.
        Si $x_1 = u$, alors $x_n = v$ et $x_1x_2\dots x_n = (uv)^k$.
        Si $x_1 = v$, alors $x_n = u$ et $x_1x_2\dots x_n = (vu)^k = (uv)^{-k}$ puisque $vu = v^{-1}u^{-1} = (uv)^{-1}$.

      \item
        \emph{Cas $n$ impair.}
        Soit $k$ tel que $n = 2k + 1$.
        Si $x_1 = u$, alors $x_1x_2\dots x_n = u(vu)^k = u(uv)^{-k}$.
        Si $x_1 = v$, alors $x_1x_2\dots x_n = v(uv)^k$.
    \end{itemize}

    Finalement,
    \[
      \Gr{u, v} = \set{(uv)^n, u(uv)^m, v(uv)^p \given (m, n, p)\in\Z\times\Z_-\times\N},
    \]
    À fortiori,
    \[
      \Gr{u, v} = \set{e, u, v, (uv)^m, u(uv)^n, v(uv)^p \given (m, n, p)\in\Z^3}.
    \]

    \item % c)
      Posons $H = \set{t, \dots, t^{k - 1}, e, u, ut, \dots, ut^{k - 1}}$.
      Bien entendu, $H\subseteq\Gr{u, v}$.
      De plus, pour tout $m\in\Z$, on a
      \[
        (uv)^m = t^m, \quad u(uv)^m = ut^m \quad\text{et}\quad v(uv)^m = u^{-1}t t^m = ut^{m + 1},
      \]
      ce qui prouve l'inclusion inverse.
      On a donc $\Gr{u, v} = H$.

      On a $(ut)^2 = (u^2v)^2 = v^2 = e$.

      Si on admet que $u\neq v$, alors $\Gr{u, v}$ est engendré par deux éléments $u$ et $t$ tels que $\ordre(u) = 2$, $\ordre(t) = k \geq 2$ et $\ordre(ut) = 2$, donc les groupes $\Gr{u, v}$ et $D_k$ sont isomorphes (remarque (30') et proposition~3.74).
  \end{enumerate}
