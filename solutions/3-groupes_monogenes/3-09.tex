\begin{enumerate}
  \item 
    Soit un entier $n > 0$.
    Notons $g^n$ la composée $g \circ g \circ \dots \circ g$ de $n$ fonctions $g$.
    Pour tout $z\in\widetilde{\C}$, on a $g^n(z) = z + n$, donc $g^n \neq \mathrm{id}_{\widetilde{\C}}$.
    Il s'ensuit que $g$ est d'ordre infini ; le sous-groupe $G$ est monogène infini, donc isomorphe à $\Z$.
  \item
    On montre facilement par récurrence que pour tout entier $n > 0$, on a $h^n(z) = \alpha^n z + \beta(1 - \alpha^n)/(1 - \alpha)$ quel que soit $z\in\widetilde{\C}$.
    Par suite, $h^n = \mathrm{id}_{\widetilde{\C}}$ si et seulement si $\alpha^n = 1$.
    Nous en déduisons que $h$ est d'ordre fini dans $\mathcal{H}$ si et seulement si $\alpha$ est une racine de l'unité dans $\C$.
\end{enumerate}
