\begin{enumerate}
  \item
    Le groupe multiplicatif des éléments inversibles de $\grq{\Z}{15\Z}$ est
    (théorème 3.18 et proposition 3.24)
    \[
      G_{15}
        = \set{\cl{k} \given 1\leq k\leq 14, \pgcd(k,15)=1}
        = \set{
            \cl{1},\cl{2},\cl{4},\cl{7},\cl{8},\cl{11},\cl{13},\cl{14}
          }.
    \]
  \item
    Soit les deux sous-groupes de $G_{15}$,
    \[
      \langle\cl{2}\rangle = \set{\cl{1},\cl{2},\cl{4},\cl{8}}
      \quad\text{et}\quad
      \langle\cl{11}\rangle = \set{\cl{1},\cl{11}}.
    \]
    Nous avons:

    \begin{enumerate}[1)]
      \item
        $\langle\cl{2}\rangle \iso C_4$ et $\langle\cl{11}\rangle \iso C_2$.
      \item
        Pour tout $h\in\langle\cl{2}\rangle$ et $k\in\langle\cl{11}\rangle$, on
        a $hk = kh$ ($G_{15}$ est un groupe abélien).
      \item
        $G_{15} = \langle\cl{2}\rangle \langle\cl{11}\rangle$ comme le montre la
        table de multiplication suivante

        \begin{center}
          \begin{tabular}{c|cccc}
            $\times$  & $\cl{1}$  & $\cl{2}$ & $\cl{4}$  & $\cl{8}$ \\
            \midrule
            $\cl{1}$  & $\cl{1}$  & $\cl{2}$ & $\cl{4}$  & $\cl{8}$ \\
            $\cl{11}$ & $\cl{11}$ & $\cl{7}$ & $\cl{14}$ & $\cl{13}$
          \end{tabular}
        \end{center}

      \item
        $\langle\cl{2}\rangle \cap \langle\cl{11}\rangle = (\cl{1})$.
    \end{enumerate}

    Donc, d'après la proposition 1.85, nous avons l'isomorphisme
    \[
      G_{15} \iso C_2 \times C_4.
    \]
\end{enumerate}
