%%% Exercice 3.32

\begin{enumerate}
  \item % a)
    La matrice $X$ est diagonale, donc pour tout entier $k\geq 0$, on a
    \[
      X^k = \matrice{\alpha^k  & 0 \\ 0 & \alpha^{-k}}.
    \]
    De plus, $\alpha^k = 1$ si et seulement si $k$ est un multiple de $n$.
    Nous en déduisons que $X$ est une matrice d'ordre $n$.
    Les matrices $Y$ et $XY = \matrice{0 & \alpha \\ \alpha^{-1} & 0}$ sont d'ordre $2$.
    Le théorème~3.74 nous permet alors de conclure que le groupe multiplicatif engendré par les matrices $X$ et $Y$ est isomorphe au groupe diédral $D_n$.

  \item % b)
    Les matrices $U$, $V$ et $UV$ sont respectivement d'ordre $4$, $2$ et $2$, donc d'après le théorème~3.74, le groupe multiplicatif engendré par les matrices $U$ et $V$ est isomorphe au groupe diédral $D_4$.
\end{enumerate}

