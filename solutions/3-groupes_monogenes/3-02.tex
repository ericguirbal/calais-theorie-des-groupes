\begin{enumerate}
  \item 
    $(\Rightarrow)$: il existe $x\in\Z$ tel que $x\equiv 0\pmod{m}$ et 
    $x\equiv 1\pmod{n}$. Nous en déduisons l'existence de deux entiers $k$ et
    $l$ tel que $x = km$ et $x - 1 = ln$. Ces deux entiers vérifient la relation
    $km - ln = 1$. D'après le théorème de Bézout, $m$ et $n$ sont premiers entre
    eux.

    $(\Leftarrow)$: d'après le théorème de Bézout, il existe deux entiers $k$ et
    $l$ tel que $km + ln = 1$. Posons $x = bkm + aln$. 
    De $bkm \equiv 0 \pmod{m}$ et $ln \equiv 1\pmod{m}$ nous déduisons
    \[
      x\equiv a\pmod{m}.
    \]
    De la même façon, $aln \equiv 0 \pmod{n}$ et $km\equiv 1\pmod{n}$ impliquent
    que
    \[
      x\equiv b\pmod{n}.
    \]

  \item
    L'application $f$ est surjective si et seulement si, pour tout
    $(a,b)\in\Z^2$, il existe $x\in\Z$ tel que 
    $\sigma(x)=\sigma(a)$ et $\pi(x)=\pi(b)$, c'est-à-dire si et seulement si pour
    tout $(a,b)\in\Z^2$, il existe $x\in\Z$ tel que $x\equiv a\pmod{m}$ et
    $x\equiv b\pmod{n}$. Le résultat de la question précédente permet de
    conclure que $f$ est surjective si et seulement si $m$ et $n$ sont premiers
    entre eux.
\end{enumerate}
