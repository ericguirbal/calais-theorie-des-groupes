%%% Exercice 3.25
À toute permutation $\sigma\in\Sym{n}$ associons l'automorphisme intérieur $f_\sigma$ définie par sur $\Sym{n}$ par $f_\sigma(\gamma) = \sigma\circ\gamma\circ\sigma^{-1}$.
L'application
\[
  f\colon\Sym{n}\to\Int(\Sym{n}), \quad \sigma\mapsto f_\sigma
\]
est un homomorphisme de groupes ; en effet, pour tout $(\sigma_1, \sigma_2)\in\Sym{n}^2$, on a
\begin{align*}
  f_{\sigma_1\circ\sigma_2}(\gamma)
    &= (\sigma_1\circ\sigma_2)\circ\gamma\circ(\sigma_1\circ\sigma_2)^{-1} \\
    &= \sigma_1\circ(\sigma_2\circ\gamma\circ\sigma_2^{-1})\circ\sigma_1^{-1} \\
    &= (f_{\sigma_1} \circ f_{\sigma_2})(\gamma).
\end{align*}
Par définition, l'homorphisme $f$ est surjectif.
Intéressons-nous à son noyau:
\begin{align*}
  \Ker(f)
    &= \set{\sigma\in\Sym{n}\given f_{\sigma} = \id_{\Sym{n}}} \\
    &= \set{\sigma\in\Sym{n}\given \sigma\circ\gamma\circ\sigma^{-1} = \gamma} \\
    &= \set{\sigma\in\Sym{n}\given \sigma\circ\gamma = \gamma\circ\sigma} \\
    & = \ZG(\Sym{n}).
\end{align*}

Démontrons par l'absurde que le centre de $\Sym{n}$ pour $n\geq 3$ est trivial.
Soient $\sigma\in\ZG(\Sym{n})\setminus\set{e}$ et $j\in\supp(\sigma)$.
Comme $n\geq 3$, il existe $k\in\N_n\setminus\set{j, \sigma(j)}$.
Notons $\tau$ la transposition $(j, k)$.
En utilisant le résultat de la question a) de l'exercice précédent, on a
\[
  (j, k) = \sigma\circ\tau\circ\sigma^{-1} = (\sigma(j), \sigma(k)).
\]
Cela implique que $j = \sigma(j)$ ou $k = \sigma(j)$ ce qui est impossible.

Nous avons montré que $f$ est un isomorphisme de groupes.
