%% Exercice 3.26

Soient $n\geq 4$ dans $\N$ et $i$, $j$, $k$, $l$ trois entiers distincts dans $\N_n$.
La décomposition
\[
  (i, j, k) = (i, j)(j, k)
\]
est triviale.
Nous en déduisons que
\[
  (i, j, k)(j, k, l) = (i, j)(j, k)(j, k)(k, l).
\]
Une transposition étant d'ordre~$2$, nous obtenons
\[
  (i, j, k)(j, k, l) = (i, j)(k, l).
\]
\begin{enumerate}
  \item % a)
    Soit $n\geq 3$ un entier.
    Les $3$-cycles de $S_n$ appartiennent à $A_n$ ;
    il suffit donc de montrer que toute permutation paire $\sigma\neq e$ de $S_n$ peut s'écrire comme un produit de $3$-cycles.
    Nous savons déjà que $\sigma$ se décompose en un produit d'un nombre pair de transpositions distinctes (théorèmes~3.60 et~3.65) :
    \[
      \gamma = \tau_1\tau_2\dots\tau_{2s} \quad (s\in\N^*).
    \]
    Regroupons les deux par deux :
    \[
      \gamma = (\tau_1\tau_2)\dots (\tau_{2s - 1}\tau_{2s}).
    \]
    Pour tout $k\in\N_s$, si $\supp(\tau_{2k - 1})\cap \supp(\tau_{2k}) = \varnothing$, alors $\tau_{2k - 1}\tau_{2k}$ est un produit de deux $3$-cycles, sinon $\tau_{2k - 1}\tau_{2k}$ est un $3$-cycle.

    On conclut que, pour tout entier $n\geq 3$, l'ensemble des $3$-cycles engendre le groupe alterné $A_n$.

  \item % b)
    Soient $i$, $j$, $k$ trois entiers distincts dans $\N_n\setminus\set{1}$ où $n\geq 3$.
    On vérifie aisément que
    \[
      (1, i, j) = (1, i)(i, j) = (1, i)(1, i)(1, j)(1, i) = (1, j)(1, i).
    \]
    Nous en déduisons la décomposition
    \begin{align*}
      (i, j, k)
        &= (i, j)(j, k) \\
        &= (1, i)(1, j)(1, i)(1, j)(1, k)(1, j) \\
        &= (1, j, i)(1, j, i)(1, j, k)
    \end{align*}
    qui permet de conclure que l'ensemble des $3$-cycles de la forme $(1, i, j)$ engendrent le groupe alterné $A_n$.

  \item % c)
    Compte tenu de ce qui précède, il suffit d'écrire un $3$-cycle comme un produit de $3$-cycles de la forme $(1, i, j)$.
    Or
    \begin{align*}
      (1, i, j) &= (1, 2, j)(1, 2, i)^2  \\
      (2, i, j) &= (1, 2, j)^2 (1, 2, i) \\
      \shortintertext{et pour tout $k\in\N_n\setminus\set{1, 2}$, }
      (k, i, j) &= (1, 2, k)(1, 2, j)^2(1, 2, i)(1, 2, k)^2,
    \end{align*}
    donc l'ensemble des $3$-cycles de la forme $(1, 2, i)$ engendrent le groupe alterné $A_n$.
\end{enumerate}
