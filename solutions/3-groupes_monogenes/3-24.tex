%%% Exercice 3.24
\begin{enumerate}
  \item
    Nous avons $r$ entiers distincts $\sigma(j_1), \sigma(j_2), \dots, \sigma(j_r)$ tels que
    \[
      \sigma\circ\gamma\circ\sigma^{-1} (\sigma(j_k))
        = \sigma\circ\gamma(j_k)
        = \begin{cases}
            \sigma(j_{k + 1})  & \text{si $1\leq k \leq r - 1$} \\
            \sigma(j_1) & \text{si $k = r$},
          \end{cases}.
    \]
    De plus, pour tout entier $k\in\N_n\setminus\set{\sigma(j_1), \sigma(j_2), \dots, \sigma(j_r)}$, il existe un unique entier $j\in\N_n\setminus\set{j_1, j_2, \dots, j_r}$ tel que $k = \sigma(j)$, et on a
    \[
      \sigma\circ\gamma\circ\sigma^{-1}(k) = \sigma\circ\gamma(j) = \sigma(j) = k.
    \]
    Nous avons prouvé que $\sigma\circ\gamma\circ\sigma^{-1}$ est le $r$-cycle $(\sigma(j_1), \sigma(j_2), \dots, \sigma(j_r))$.

  \item
    Pour tout entier $p$ tel que $1\leq p\leq n$, on a
    \[
      \gamma_1^p =
        \matrice{
          1 & 2 & \dots & n - p & n -p + 1 & \dots & n \\
          p + 1 & p + 2 & \dots & n & 1 & \dots & p
        }.
    \]
    De la question précédente, nous déduisons que $\gamma_1^p\circ\tau_1\circ\gamma_1^{-p}$ est la transposition
    \[
        (\gamma_1^p(1), \gamma_1^p(2))
        = \begin{cases}
            (p + 1, p + 2) & \text{si $1\leq p\leq n - 2$} \\
            (1, n) & \text{si $p = n - 1$} \\
            (1, 2) & \text{si $p = n$}.
          \end{cases}
    \]

    Remarquons que
    \[
      \set{(i, i + 1)\given 1\leq i\leq n - 1}\subseteq\set{\gamma_1^p\circ\tau_1\circ\gamma_1^{-p}\given 1\leq p\leq n} \subseteq \Gr{\tau_1, \gamma_1}\subseteq\Sym{n}.
    \]
    Or l'ensemble des transpositions $(i, i + 1)$ telles que $1\leq i\leq n - 1$ engendrent le groupe $\Sym{n}$ (proposition~3.64), d'où la conclusion:
    \[
      \Sym{n} = \Gr{\tau_1, \gamma_1}.
    \]
\end{enumerate}
