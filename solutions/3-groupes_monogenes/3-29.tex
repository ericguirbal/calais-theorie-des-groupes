\begin{enumerate}
  \item  % a)
    La commutativité de $\pi$ avec $\sigma$ résulte de la commutativité de deux cycles disjoints.
    On montre que
    \[
      \pi\circ\tau = \matrice{
        1 & 2 & 3 & 4 & 5 & 6 & 7 & 8 & 9 \\
        5 & 6 & 4 & 8 & 9 & 7 & 2 & 3 & 1
      } = \tau\circ\pi.
    \]

  \item % b)
    Les permutations $\pi$, $\sigma$ et $\tau$ sont d'ordre $3$.
    Décomposons $\sigma\circ\tau$ en produit de cycles disjoints:
    %
    \begin{align*}
      \sigma\circ\tau 
        &= \matrice{
            1 & 2 & 3 & 4 & 5 & 6 & 7 & 8 & 9 \\
            5 & 6 & 4 & 8 & 9 & 7 & 1 & 2 & 3
        } \\
        &= (1,5,9,3,4,8,2,6,7).
    \end{align*}
    %
    Nous en déduisons que la permutation $\sigma\circ\tau$ est d'ordre $9$.

  \item % c)
    On a
    %
    \begin{align*}
      \tau\circ\sigma\circ\tau^{-1}
        &= (1,4,7)(2,5,8)(3,6,9)(4,5,6)(7,8,9)(1,7,4)(2,8,5)(3,9,6) \\
        &= \matrice{
            1 & 2 & 3 & 4 & 5 & 6 & 7 & 8 & 9 \\
            2 & 3 & 1 & 4 & 5 & 6 & 8 & 9 & 7 
        } \\
        &= (1,2,3)(7,8,9),
    \end{align*}
    %
    et
    %
    \begin{align*}
      \tau^{-1}\circ\sigma\circ\tau
        &= (1,7,4)(2,8,5)(3,9,6)(4,5,6)(7,8,9)(1,4,7)(2,5,8)(3,6,9) \\
        &= \matrice{
            1 & 2 & 3 & 4 & 5 & 6 & 7 & 8 & 9 \\
            2 & 3 & 1 & 5 & 6 & 4 & 7 & 8 & 9
        } \\
        &= (1,2,3)(4,5,6).
    \end{align*}
    %
    Enfin, s'il existe un entier $k$ tel que $\pi = (\sigma\circ \tau)^k$, on peut prendre $k\in\set{1,\dots,8}$ et
    d'après l'exercice III.13, $9/(9,k) = 3$, d'où $k = 3$ ou $k = 6$.
    On vérifie immédiatement que $k = 6$ convient.
\end{enumerate}
