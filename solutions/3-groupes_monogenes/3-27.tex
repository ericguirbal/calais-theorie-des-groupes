Posons $\sigma_1 = (1,2)(3,4)$, $\sigma_2 = (1,3)(2,4)$ et $\sigma_3 = (1,4)(2,3)$.
Chacune de ses permutations, ainsi que $e$, ont une signature égale à $1$, donc
$K\subseteq \Alt{4}$.
La table de multiplication ci-dessous est un carré latin, ce qui prouve que $K$
est un sous-groupe de $\Alt{4}$.

\begin{center}
  \begin{tabular}{c|cccc}
        $\times$ &        $e$ & $\sigma_1$ & $\sigma_2$ & $\sigma_3$ \\
      \midrule
             $e$ &        $e$ & $\sigma_1$ & $\sigma_2$ & $\sigma_3$ \\
      $\sigma_1$ & $\sigma_1$ &        $e$ & $\sigma_3$ & $\sigma_2$ \\
      $\sigma_2$ & $\sigma_2$ & $\sigma_3$ &        $e$ & $\sigma_1$ \\
      $\sigma_3$ & $\sigma_3$ & $\sigma_2$ & $\sigma_1$ &        $e$
  \end{tabular}
\end{center}

Nous reconnaissons la table de multiplication du groupe de Klein, donc le groupe $K$ est isomorphe au groupe de Klein.
