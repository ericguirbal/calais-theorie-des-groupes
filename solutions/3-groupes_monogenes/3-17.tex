%%% Exercice 3.17

\begin{enumerate}
  \item % a)
    Soit $\alpha\in\Aut(G)$.
    On a
    \[
      \Gr{\alpha(x)}
        = \set{\alpha(x)^n\given n\in\Z}
        = \set{\alpha(x^n)\given n\in\Z}
        = \alpha(G)
        = G,
    \]
    ce qui prouve que $\alpha(x)$ est un générateur de $G$.

  \item % b)
    \emph{L'application $\lambda_k$ est un automorphisme de $G$.}
    Pour tout entier $k$ premier avec $n$, notons $\lambda_k$ l'application notée $\lambda$ dans l'énoncé.
    Dans le groupe abélien $G$, on a $(ab)^k = a^k b^k$ pour tout $(a, b)\in G^2$ ; par conséquent, l'application $\lambda_k$ est un homomorphisme de groupes.

    Soit $a\in\Ker(\lambda_k)$.
    Il existe $l\in\Z$ tel que $a = x^l$.
    Alors $a^k = e$ implique $x^{kl} = e$, d'où $kl\in n\Z$ (proposition~3.6).
    Comme $k$ et $n$ sont premiers entre eux, on a $l\in n\Z$ (lemme de Gauss), donc $a = e$ (proposition~3.6), prouvant ainsi que l'homomorphisme $\lambda_k$ est injectif.
    Puisque $G$ est fini, l'homomorphisme $\lambda_k$ est aussi surjectif.
    Finalement $\lambda_k$ est un automorphisme de $G$.

    \emph{Les groupes $\Aut(G)$ et $G_n$ sont isomorphes.}
    Montrons que $\lambda_k = \lambda_l$ si et seulement si $\cl{k} = \cl{l}$.
    Supposons que $\lambda_k = \lambda_l$.
    Alors $x^k = x^l$, puis $x^{k - l} = e$, donc $k - l\in n\Z$, soit $\cl{k} = \cl{l}$.
    Réciproquement, si $\cl{k} = \cl{l}$, alors $k - l\in n\Z$, puis $a^{k - l} = e$ (corollaire~2.10), si bien que $a^k = a^l$, c'est-à-dire $\lambda_k = \lambda_l$.

    Ce résultat montre que la correspondance
    \begin{align*}
      \psi\colon G_n &\to \Aut(G) \\
                \cl{k} &\mapsto \lambda_k.
    \end{align*}
    définit une application injective.
    Elle est également surjective ; en effet, si $\lambda\in\Aut(G)$, alors il existe un entier $k$ premier avec $n$ tel que $\lambda(x) = x^k$ (question précédente et théorème~3.18 2\up{o}) et pout tout $i\in\Z$, on a
    \[
      \lambda(x^i) = \lambda(x)^i = x^{ki} = (x^i)^k = \lambda_k(x^i),
    \]
    donc $\lambda = \psi(\cl{k})$.

    L'application $\psi$ est un homomorphisme de groupes ; en effet, pour tout $(\cl{k}, \cl{l})\in G_n^2$ et tout $a\in G$, on a
    \[
      (\lambda_k\circ\lambda_l)(a)
        = \lambda_k\bigl(\lambda_l(a)\bigr)
        = \lambda_k(a^l)
        = a^{kl}
        = \lambda_{kl}(a),
    \]
    c'est-à-dire
    \[
      \psi(\cl{k}\cl{l}) = \psi(\cl{k})\circ\psi(\cl{l}).
    \]
    On conclut que les groupes $G_n$ et $\Aut(G)$ sont isomorphes.
    Par conséquent, le groupe $\Aut(G)$ est abélien comme $G_n$ et $\card{\Aut(G)} = \varphi(n)$ (proposition~3.24).

  \item % c)
    Soit $\lambda\in\Aut(G)$.
    On sait que $\lambda(x)$ est un générateur de $G$, or les seuls générateurs du groupe monogène $G$ sont $x$ et $x^{-1}$ (théorème~3.18), donc $\Aut(G)$ est un groupe d'ordre~$2$, dont les éléments sont l'identité $\id_G$ et l'application inverse $G\to G$, $a\mapsto a^{-1}$.
    La proposition~3.9 nous permet de conclure que $\Aut(G)$ est un groupe cyclique.

  \item % d)
    Soit $p$ un nombre premier.
    Les groupes $\Z$ et $\Zn{p}$ ne sont pas isomorphes.
    Cependant, d'après ce qui précède, on a $\Aut(\Z) = 2$ et $\Aut(\Zn{p}) = \varphi(p) = 2$, donc $\Aut(\Z)\iso\Aut(\Zn{p})$.

    De même, si $q$ est nombre premier distinct de $p$, les groupes $\Zn{p}$ et $\Zn{q}$ ne sont pas isomorphes, alors que $\Aut(\Zn{p})$ et $\Aut(\Zn{q})$ le sont.
\end{enumerate}
