%%% Exercice 5-13
Faisons une démonstration par l'absurde.
Supposons que $\ZG(G) = (e)$.
Soient $\set{x_1,\dots,x_{k - 1}}$ une famille de représentants des classes de conjuguaison distinctes et non ponctuelles de $G$.
D'après l'équation aux classes, nous avons
\[
  \ordre(G) = 1 + \sum_{i = 1}^{k - 1} \card{\Omega_{x_i}}.
\]
Comme $\card{\Omega_{x_i}}$ divise $\ordre(G)$ pour tout $i\in\set{1,\dots,k - 1}$, on a $\card{\Omega_{x_i}} \geq p$.
Par conséquent, $\ordre(G) \geq 1 + (k - 1)p$, puis
\[
  \frac{\ordre(G)}{p} \geq \frac{1}{p} + k - 1 > k - 1,
\]
d'où la contradiction
\[
  \frac{\ordre(G)}{p} \geq k.
\]
Nous avons donc bien $\ZG(G) \neq (e)$.
