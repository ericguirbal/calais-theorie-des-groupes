%%% Exercice 5-22
Soit $f$ un isomorphisme de $N_1$ sur $N_2$.
Nous savons que
%
\begin{align*}
  f_\sharp\from\Aut(N_1) &\to \Aut(N_2) \\
  g &\mapsto f\circ g\circ f^{-1}
\end{align*}
%
est un isomorphisme de groupes (exercice 32, chapitre I).
Considérons l'application
\begin{align*}
  \varphi\from\Hol(N_1) &\to\Hol(N_2) \\
    (x, \alpha) &\to (f(x), f_\sharp(\alpha)).
\end{align*}

L'application $\varphi$ est un morphisme de groupes ; en effet, quels que soient  $g = (x, \alpha)$ et $h = (y, \beta)$ dans $\Hol(N_1)$, on a
\begin{align*}
  \varphi(g)\varphi(h)
    &= \bigl( f(x), f_\sharp(\alpha) \bigr) \bigl( f(y), f_\sharp(\beta) \bigr) \\
    &= \bigl( f(x) f_\sharp(\alpha)(f(y)), f_\sharp(\alpha)\circ f_\sharp(\beta) \bigr) \\
    &= \bigl( f(x) (f(\alpha(y)), f_\sharp(\alpha\circ\beta) \bigr) \\
    &= \bigl( f(x\alpha(y)), f_\sharp(\alpha\circ\beta) \bigr) \\
    &= \varphi(x\alpha(y), \alpha\circ\beta) \\
    &= \varphi\bigl( (x, \alpha)(y, \beta) \bigr) \\
    &= \varphi(gh).
\end{align*}
%
L'application $\varphi$ est bijective, car, $\ker \varphi = \ker f  \times \ker f_\sharp = (e, \id_{N_1})$ et $\im \varphi = \im f \times \im f_\sharp = N_2\times\Aut(N_2)$.

En conclusion, les groupes $\Hol(N_1)$ et $\Hol(N_2)$ sont isomorphes.
