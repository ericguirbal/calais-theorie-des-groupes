\begin{enumerate}
  \item 
    Soit $G = \Gr{(1,2,3)} = \set{e, (1,2,3), (1,3,2)}$.
    Nous avons une partition de $\N_4$ en exactement deux $G$-orbites: $\Omega_1 = \set{1,2,3}$ et $\Omega_4 = \set{4}$.
    Les stabilisateurs de l'élément $i\in\N_4$ sont les permutations $\sigma\in G$ telles que $i\notin\supp(\sigma)$, d'où $G_1 = G_2 = G_3 = \set{e}$ et $G_4 = G$.

  \item 
    Soit $G = \Gr{(1,2),(3,4)} = \set{e, (1,2),(3,4),(1,2)(3,4)}$.
    Le $G$-ensemble $\N_4$ a deux orbites, $\Omega_1 = \set{1,2}$ et $\Omega_3 = \set{3,4}$, et les stabilisateurs sont $G_1 = G_2 = \set{e,(3,4)}$ et $G_3 = G_4 = \set{e, (1,2)}$.

  \item
    Soit $G = \Alt{4}$.
    C'est un groupe d'ordre $12$ dont les éléments sont $e$, les produits de deux transpositions à supports disjoints et les $3$-cycles.
    Déterminons l'orbite de $1$.
    Comme $G$ contient les permutations $(1,2)(3,4)$, $(1,3)(2,4)$ et $(1,4)(2,3)$, nous en déduisons que  $\Omega_1 = \N_4$, et donc qu'il n'y a qu'une seule $G$-orbite.
    Les stabilisateurs sont $G_1 = \set{e, (2,3,4), (2,4,3)}$, $G_2 = \set{e, (1,3,4), (1,4,3)}$, $G_3 = \set{e, (1,2,4), (1,4,2)}$ et $G_4 = \set{e, (1,2,3), (1,3,2)}$.
\end{enumerate}
