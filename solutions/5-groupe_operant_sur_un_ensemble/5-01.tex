Pour tous nombres réels $a$, $b$ et $c$, notons $M(a,b,c)$ la matrice
\[
  \matrice{a & b \\ 0 & c},
\]
de sorte que
\[
  G = \set[\big]{M(a,b,c) \given ac\neq 0}.
\]
On a $\det(M(a,b,c)) = ac$, donc $G\subseteq GL(2,\R)$.
L'ensemble $G$ est non vide, car il contient la matrice identité $I = M(1,0,1)$.
Soient $M(a,b,c)$ et $M(a',b',c')$ deux matrices de $G$.
Alors
\[
  M(a,b,c)^{-1}
    = \frac{1}{ac}\matrice{c & -b \\ 0 & a}
    = M\Bigl( \frac{1}{a}, -\frac{b}{ac}, \frac{1}{c} \Bigr)
    \in G
\]
et
\[
  M(a,b,c) M(a',b',c')
    = \matrice{aa' & ab' + bc' \\ 0 & cc'}
    = M(aa', ab'+bc', cc')
    \in G
\]
donc $G$ est un sous-groupe de $\GL(2,\R)$.

L'application
%
\begin{align*}
     G\times\R \to \R, \quad \bigl(M(a,b,c),x\bigr) \mapsto M(a,b,c) \cdot x =  \frac{ax+b}{c}
\end{align*}
%
définit une action de $G$ sur $\R$ car elle vérifie les deux conditions suivantes:
%
\begin{itemize}
  \item pour tout $x\in\R$, on a $I\cdot x = x$;
  \item pour toutes matrices $M(a,b,c)$ et $M(a',b',c')$ de $G$, on a
    %
    \begin{align*}
      M(a,b,c) \cdot \bigl( M(a',b',c') \cdot x \bigr)
        &= M(a,b,c) \cdot \frac{a'x + b'}{c'} \\
        &= \frac{a\frac{a'x + b'}{c'}  + b}{c} \\
        &= \frac{aa'x + ab' + bc'}{cc'} \\
        &= M(aa', ab'+bc', cc') \cdot x \\
        &= \bigl( M(a,b,c)M(a',b',c') \bigr) \cdot x.
    \end{align*}
\end{itemize}

Le noyau de cette action est le sous-groupe
%
\[
  \begin{split}
    \bigl\{\, M(a,b,c)\in &G\:;\:M(a,b,c) \cdot x = x \text{ pour tout $x\in\R$} \,\bigr\} \\
    &= \set[\big]{M(a,b,c)\in G \given (ax + b)/c = x \text{ pour tout $x\in\R$}} \\
    &= \set[\big]{M(a,b,c)\in G \given (a - c)x = -b \text{ pour tout $x\in\R$}} \\
    &= \set[\big]{M(a,b,c)\in G \given a = c \text{ et $b = 0$}} \\
    &= \set[\big]{aI \given a\in\R^*}.
  \end{split}
\]

Le stabilisateur de 0 est le sous-groupe
%
\begin{align*}
  G_0
    &= \set[\big]{M(a,b,c)\in G \given M(a,b,c) \cdot 0 = 0 } \\
    &= \set[\big]{M(a,b,c)\in G \given b/c = 0 } \\
    &= \set[\big]{M(a,0,c)\in G},
\end{align*}
%
c'est-à-dire le sous-groupe de $G$ des matrices diagonales.

L'orbite de $0$ est l'ensemble
%
\begin{align*}
  \Omega_0
    &= \set[\big]{ M(a,b,c) \cdot 0 \given M(a,b,c)\in G } \\
    &= \set[\big]{ b/c \given (b,c)\in\R\times\R^*} \\
    &= \R.
\end{align*}

