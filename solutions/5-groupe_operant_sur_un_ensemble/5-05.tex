Pour toute matrice $M = \matrice{a & 0 \\ 0 & b}\in\Gamma$, on a
\[
  (0,0) M = (0,0),
  \quad
  (1,1) M = (a,b),
  \quad\text{et}\quad
  (0,1) M = (0,b),
\]
%
d'où nous déduisons immédiatement que 
\[
  \Gamma_{(0,0)} = \Gamma,
  \quad
  \Gamma_{(1,1)} = I
  \quad\text{et}\quad
  \Gamma_{(0,1)} = \set*{\matrice{a & 0 \\ 0 & 1} \given a > 0}.
\]

Les $9$ $\Gamma$-orbites sont:

\begin{itemize}
  \item
    les quatre quadrants
    %
    \begin{align*}
      \Gamma_{(1,1)}   &= \set{(x,y)\in\R^2 \given x > 0, y > 0}, \\
      \Gamma_{(-1,1)}  &= \set{(x,y)\in\R^2 \given x < 0, y > 0}, \\
      \Gamma_{(-1,-1)} &= \set{(x,y)\in\R^2 \given x < 0, y < 0}, \\
      \Gamma_{(1,-1)}  &= \set{(x,y)\in\R^2 \given x > 0, y < 0};
    \end{align*}

  \item
    les quatre demi-droites
    %
    \begin{align*}
      \Gamma_{(1,0)} &= \set{(x,y)\in\R^2 \given x > 0, y = 0}, \\
      \Gamma_{(0,1)} &= \set{(x,y)\in\R^2 \given x = 0, y > 0}, \\
      \Gamma_{(-1,0)} &= \set{(x,y)\in\R^2 \given x < 0, y = 0}, \\
      \Gamma_{(0,-1)} &= \set{(x,y)\in\R^2 \given x = 0, y < 0}; \\
    \end{align*}

  \item
    le point $\Gamma_{(0,0)} = {(0,0)}$.
\end{itemize}
