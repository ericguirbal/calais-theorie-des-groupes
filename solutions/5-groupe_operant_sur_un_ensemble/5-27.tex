%%% Exercice 5.27

\begin{enumerate}[label=\arabic*\up{o}]
  \item % 1°
    \begin{enumerate}
      \item % 1° a)
        Soit $x\in E$.
        Étant donné que $\theta$ est un isomorphisme, il nous suffit de démontrer que $\theta(G_x) = G'_{\lambda(x)}$.
        Quel que soit $g\in G_x$, on a $\theta(g)\cdot\lambda(x) = \lambda(g\cdot x) = \lambda(x)$, donc $\theta(G_x)\subseteq G'_{\lambda(x)}$.
        Réciproquement, si $g'\in G'_{\lambda(x)}$, il existe $g\in G$ tel que $\theta(g) = g'$.
        On a alors $\lambda(g\cdot x) = \theta(g)\cdot\lambda(x) = g'\cdot\lambda(x) = \lambda(x)$, si bien que $g\cdot x = x$, donc $G'_{\lambda(x)} \subseteq \theta(G_x$).

      \item % 1° b)
        \emph{$G \text{ transitif sur } E \iff G' \text{transitif sur } E'$.}
        Soit $x\in E$.
        L'orbite de $x$ étant notée $\Omega_x$ et celle de $\lambda(x)$ étant notée $\Omega'_{\lambda(x)}$, on a
        \begin{align*}
          \Omega'_{\lambda(x)}
            &= \set{g'\cdot\lambda(x)\given g'\in G'} \\
            &= \set{\theta(g)\cdot\lambda(x)\given g\in G} \\
            &= \set{\lambda(g\cdot x)\given g\in G} \\
            &= \lambda(\Omega_x).
        \end{align*}
        Nous en déduisons immédiatement que $G$ est transitif sur $E$ si et seulement si $G'$ est transitif sur $E'$.

        \emph{$G \text{ transitif sur } E \implies \forall (x, y)\in E\times E', G_x\iso G'_y$.}
        Soit $(x, y)\in E\times E'$.
        D'après la question précédente, les sous-groupes $G_x$ et $G'_{\lambda(x)}$ sont isomorphes.
        Nous venons de démontrer que $G'$ est transitif sur $E'$, donc il existe $g'\in G'$ tel que $y = g'\cdot\lambda(x)$.
        Il s'ensuit que $G'_y$ et $G'_{\lambda(x)}$ sont isomorphes (théorème~5.18).
        Finalement, les sous-groupes $G'_y$ et $G_x$ sont isomorphes.

      \item % 1° c)
        Soient les sous-groupes $H = \cap_{x\in E} G_x$ et $H' = \cap_{x\in E'} G_{x'}$.
        Remarquons que le groupe $G$ opère fidèlement sur $E$ si et seulement si $H = (e)$.
        Comme $\theta$ est injective, on a
        \[
          \theta(H) = \bigcap_{x\in E} \theta(G_x),
        \]
        puis d'après la question~\primo a),
        \[
          \theta(H) = \bigcap_{x\in E} G'_{\lambda(x)},
        \]
        et enfin, $\lambda$ étant surjective,
        \[
          \theta(H) = \bigcap_{x'\in E'} G_{x'} ;
        \]
        autrement dit,
        \[
          \theta(H) = H'.
        \]
        Étant donné que $\theta$ est un monomorphisme, nous en déduisons immédiatement que $G$ opère fidèlement sur $E$ si et seulement si $G'$ opère fidèlement sur $E'$.
    \end{enumerate}
  \item % 2°
    Pour tout $(g, h)\in G^2$, on a
    \begin{align*}
      \varphi(gh)
        &= \lambda\circ\gamma_{gh}\circ\lambda^{-1} \\
        &= \lambda\circ\gamma_g\circ\gamma_h\circ\lambda^{-1} \\
        &= \lambda\circ\gamma_g\circ\lambda^{-1}\circ\lambda\circ\gamma_h\lambda^{-1} \\
        &= \varphi(g)\circ\varphi(h).
    \end{align*}
    De plus,
    \begin{align*}
      \Ker\varphi
        &= \set{g\in G\given \lambda\circ\gamma_g\circ\lambda^{-1} = \id_{E'}} \\
        &= \set{g\in G\given \lambda_g = \lambda^{-1}\circ\id_{E'}\circ\lambda} \\
        &= \set{g\in G\given \lambda_g = \id_E} \\
        \shortintertext{et puisque $G$ agit fidèlement sur $E$,}
        \Ker\varphi &= \set{e}.
    \end{align*}
    Nous avons prouvé que $\varphi$ est un monomorphisme de groupes.

    Il s'ensuit que la corestriction de $\varphi$ à $G'$, que nous noterons $\hat{\varphi}$, est un isomorphisme de groupes.
    Soit $\gamma'$ le morphisme de groupes de $G'$ dans $\Sym{E'}$ associé à l'action naturelle de $G'$ sur $E'$.
    Il est définie par $\gamma'_{g}(x) = g(x)$ pour tout $(g, x)\in G'\times E'$.
    Pour tout $(g, x)\in G\times E$, on a
    \begin{align*}
      \hat{\varphi}(g)\cdot\lambda(x)
        &= \hat{\varphi}(g)(\lambda(x)) \\
        &= (\lambda\circ\gamma_g\circ\lambda^{-1})(\lambda(x)) \\
        &= \lambda(\gamma_g(x)) \\
        &= \lambda(g\cdot x),
    \end{align*}
    ce qui nous permet de conclure que l'action de $G$ sur $E$ est équivalente à l'action naturelle de $G'$ sur $E'$.

\end{enumerate}
