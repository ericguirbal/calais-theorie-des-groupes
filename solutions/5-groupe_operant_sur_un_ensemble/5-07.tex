\begin{enumerate}
  \item 

    Soient $(x,y)\in\N_n^2$.
    Étant donné que $G$ opère transitivement sur $\N_n$, il existe une permutation $\pi\in G$ telle que $\pi(x) = y$.
    Soit $\sigma\in H_x$.
    Comme $H\normal G$, on a $\pi\circ\sigma\circ\pi^{-1}\in H$.
    De plus, $(\pi\circ\sigma\circ\pi^{-1})(y) = (\pi\circ\sigma)(x) = \pi(x) = y$, donc $\pi\circ\sigma\circ\pi^{-1}\in H_y$.
    Il s'ensuit que $\pi H_x \pi^{-1} \subseteq H_y$, d'où $\ordre(H_x) \leq \ordre(H_y)$.
    Par symétrie, l'inégalité opposée est également vraie, donc $\ordre(H_y) = \ordre(H_x)$.
    La conclusion vient du théorème~5.19 et de la proposition~2.15:
    \[
      \card{\Omega_x} 
        = \ordre(H)/\ordre(H_x) 
        = \ordre(H)/\ordre(H_y)
        = \card{\Omega_y}.
    \]

  \item

    Supposons que $H = (e)$, ce que l'énoncé n'exclut pas.
    Les $H$-orbites de $N_n$ sont donc toutes ponctuelles.
    Puisque $n\geq 2$, nous en déduisons que $H$ n'est pas transitif sur $\N_n$.

    Supposons maintenant que $H \neq (e)$.
    Puisque les $H$-orbites forment une partition de $\N_n$, le nombre d'orbites $r$, leur cardinal $m$ ainsi que $n$ sont liés par la relation $rm = n$.
    Étant donné que $n$ est premier, on a $r = 1$.
    Autrement dit, le groupe $H$ opère transitivement sur $N_n$.
\end{enumerate}
