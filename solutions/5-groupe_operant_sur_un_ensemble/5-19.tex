%%% Exercice 5-19
\begin{enumerate}
  \item % a)
    Évident.

  \item % b)
    Supposons que $G$ est $k$-transitif sur $E$.
    Soient $x\in E$ et deux $(k - 1)$-uplets $(x_1, x_2, \dots, x_{k - 1})$, $(y_1, y_2, \dots, y_{k - 1})$ d'éléments distincts de $E \setminus\set{x}$.
    Alors $(x, x_1, \dots, x_{k - 1})$ et $(x, y_1, \dots, y_{k - 1})$ sont deux $k$-uplets d'éléments distincts de $E$, donc il existe $g\in G$ tel $g\cdot x = x$ et $g\cdot x_i = y_i$ pour tout $i\in\N_{k - 1}$, ce qui nous permet de conclure que $G_x$ est $(k - 1)$-transitif sur $E\setminus\set{x}$.

    Réciproquement, supposons que pour tout $x\in E$, le groupe $G_x$ est $(k - 1)$-transitif sur $E\setminus\set{x}$.
    Soient $(x_1, x_2, \dots, x_k)$ et $(y_1, y_2, \dots, y_k)$ deux $k$-uplets d'éléments distncts de $E$.
    Il existe $(g, h)\in G_{x_k}\times G_{y_{k - 1}}$ tel que $g\cdot x_i = y_i$ pour tout $i\in\N_{k - 1}$, $h\cdot y_i = y_i$ pour tout $i\in\N_{k - 2}$ et $h\cdot x_k = y_k$.
    On vérifie que
    \[
      hg\cdot x_i = h\cdot g\cdot x_i =
        \begin{cases}
          h\cdot y_i = y_i & \text{si $i\in\N_{k - 1}$} \\
          h\cdot x_k = y_k & \text{si $i = k$}.
        \end{cases}
    \]
    Le groupe $G$ est bien $k$-transitif sur $E$.

  \item % c)
    \begin{itemize}
      \item
        D'après la question a), $G$ est transitif sur $E$, donc la formule de Burnside (exercice~5.17) s'écrit
        \[
          \card{G} = \sum_{g\in G} \card{F(g)}.
        \]
        Soit $x\in E$.
        Le groupe $G_x$ opère transitivement sur $E\setminus\set{x}$ ; les $G_x$-orbites de $E$ sont donc $E\setminus\set{x}$ et $\set{x}$.
        La formule établie à l'exercice~18 donne
        \[
          2\card{G} = \sum_{g\in G} \card{F(g)}^2.
        \]
      \item
        Notons $E^{(k)}$ l'ensemble de $k$-uplets d'éléments deux à deux distincts de $E$.
        L'action de $G$ sur $E$ induit une action de $G$ sur $E^{(k)}$ definie par
        \[
          g\cdot(x_1, x_2, \dots, x_k) = (g\cdot x_1, g\cdot x_2, \dots, g\cdot x_k)
        \]
        pour tout $g\in G$ et $(x_1, x_2, \dots x_k)\in E^{(k)}$.
        L'action de $G$ sur $E$ étant $k$-transitive, celle de $G$ sur $E^{(k)}$ est transitive.
        Par conséquent, nous déduisons du théorème~5.19 que
        \[
          \card{G} = \card{G_x} \card{E^{(k)}}
        \]
        où $x\in E^{(k)}$.
        Or $\card{E^{(k)}} = n(n - 1)\dots (n - k + 1)$, ce qui prouve que $\card{G}$ est divisible par $n(n - 1)\dots (n - k + 1)$.

    \end{itemize}

\end{enumerate}
