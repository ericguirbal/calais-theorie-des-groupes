Du fait que tout élément de $G$ est simplifiable à gauche et à droite, les
applications translation à gauche et translation à droite sont injectives.
Comme en plus $G$ est fini, elles sont bijectives. Autrement dit, pour tout
$(a,b)\in G^2$, chacune des équations $ax = b$ et $xa = b$ d'inconnue $x$
possède une unique solution.

Montrons l'existence d'un élément neutre. Soit $a\in G$. Il existe $g\in G$ tel
que $ga = a$. Vérifions que $g$ est un neutre à gauche. Soit $x\in G$. Il
existe $h\in G$ tel que $x = ah$. On a alors $gx = g(ah) = (ga)h = ah = x$, ce
qui prouve que $g$ est un neutre à gauche. De même, on montre l'existence
d'un neutre à droite $g'$. Enfin on a $gg' = g'$ car $g$ est un neutre à
gauche, et $gg' = g$ car $g'$ est un neutre à droite, d'où $g = g'$. Notons $e$
l'élément neutre.

Il reste à montrer que tout élément de $G$ est inversible. Soit $x\in G$. Il
existe $y\in G$ tel que $xy = e$.  Donc $y(xy) = y$ soit $(yx)y = y$.
Multiplions membre à membre par l'inverse à droite de $y$. Nous obtenons $yx =
e$. Donc $x$ est inversible. 

Finalement $(G,\cdot)$ est un groupe.
