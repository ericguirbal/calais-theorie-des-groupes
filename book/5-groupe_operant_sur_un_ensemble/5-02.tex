\begin{enumerate}
  \item
    Soit $\set{x_i}_{1\leq i\leq r}$ une famille de représentants des
    $G$-orbites non ponctuelles. Puisque $E_G=\emptyset$, on a l'égalité
    \[
      17 = \sum_{i=1}^r \card{\Omega_{x_i}}.
    \]
    Pour chaque $i\in\set{1,\dots,r}$, $\card{\Omega_{x_i}}$ divise
    $\card{G}=15$, donc $\card{\Omega_{x_i}}\in\set{3,5,15}$. Notons $\alpha$
    (resp. $\beta$, $\gamma$) le nombre de $G$-orbites de longueur $3$ (resp.
    $5$, $15$). Le triplet $(\alpha,\beta,\gamma)$ vérifie l'équation
    \[
      17 = 3x + 5y + 15z
    \]
    dont l'unique solution dans $\N^3$ est $(4,1,0)$. Nous concluons qu'il y a
    $5$ $G$-orbites: $4$ $G$-orbites de longueur $3$ et une $G$-orbite de
    longueur $5$.
  
  \item 
    Soit $\set{x_i}_{1\leq i\leq r}$ une famille de représentants des 
    $G$-orbites non ponctuelles. Supposons que $E_G$ soit vide. On a alors
    \[
      19 = \sum_{i=1}^r \card{\Omega_{x_i}}.
    \]
    Pour chaque $i\in\set{1,\dots,r}$, $\card{\Omega_{x_i}}$ divise
    $\card{G}=33$, donc $\card{\Omega_{x_i}}\in\set{3,11,33}$. Soit $\alpha$
    (resp. $\beta$, $\gamma$) le nombre de $G$-orbites de longueur~$3$ (resp.
    $11$, $33$). Le triplet $(\alpha,\beta,\gamma)$ est solution de l'équation
    \[
      19 = 3x + 11y + 33z.
    \]
    Or, nous pouvons facilement vérifier que cette équation n'a pas de solution
    dans $\N^3$. Contradiction. Nous en déduisons que l'ensemble $E_G$ des
    points fixes n'est pas vide.
\end{enumerate}

