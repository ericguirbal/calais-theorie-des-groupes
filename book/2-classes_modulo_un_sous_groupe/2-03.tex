\begin{enumerate}[a)]
  \item Soit $(i,j)\in\{1,\dots,n\}^2$ tel que $Hx_i\cap Hx_j\neq\emptyset$ et
    soit $z\in Hx_i\cap Hx_j$. Alors il existe $(h,h')\in H^2$ tel que
    $z=hx_i=h'x_j$ d'où $h'^{-1}h=x_jx_i^{-1}\in H\cap K$. Ainsi $(H\cap
    K)x_i=(H\cap K)x_j$, donc $i=j$.  
    
    On sait déjà que $HK\supseteq\bigcup_{i=1}^{n} Hx_i$. Soit $z=hk\in HK$ avec
    $(h,k)\in H\times K$.  Il existe $i\in \{1,\dots,n\}$ tel que $k\in
    (H\cap K)x_i\subseteq Hx_i$ donc $z=hk\in Hx_i$ et
    $HK=\bigcup_{i=1}^n Hx_i$. 
    
    En conclusion, $\{Hx_i\}_{1\leqslant i\leqslant n}$ est une partition de
    $HK$.

  \item On a $HK=\bigcup_{i=1}^{n} Hx_i$ avec $Hx_i\cap Hx_j=\emptyset$ si
    $i\neq j$ et $|Hx_i|=|H|$. Donc $|HK|=n|H|=[K\colon H\cap K]|H|$ puis
    $|HK|=\dfrac{|K||H|}{|H\cap K|}=|KH|$.

  \item $H, K$ sont des sous-groupes du groupe fini $HK$. Nous pouvons donc
    appliquer le résultat b) de l'exercice 2: 
    $[HK\colon H\cap K]=[HK\colon H][HK\colon K]$ c'est-à-dire 
    $\dfrac{|HK|}{|H\cap K|}=\dfrac{|HK|}{|H|}\dfrac{|HK|}{|K|}$ 
    et enfin $|KH|=|HK|=\dfrac{|H||K|}{|H\cap K|}$.
\end{enumerate}

